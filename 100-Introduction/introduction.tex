%%
%% Chapter: 1
%%
\chapter{Introduction}
\label{cha:Introduction}      %% No special characters, no space character

%Remove explanations and start with text here. Leave sections and subsections.
\emph{The introduction should "set the stage" for which the work is undertaken. It should go from the general to the specific; that is, begin with a general description of the problem or technology area and proceed to the particular issue that is being addressed. It is not a summary of the work but rather the background and rationale for doing it. It shall be 3-5 pages.}

\section{Background}
\emph{Background information and events to acquaint the reader with the purpose for carrying out the work (e.g. production difficulties, redesign, design problems, material selection, excessive deflections, etc.).
General references to other work (newspapers, journals etc) that help define the problem.}

\section{Problem Statement}
\emph{An accurate and concise statement of the problem providing details not included in the abstract that leads into the body of the report.}

\section{Objectives}
\emph{Main objective and the specific objectives to be achieved and scope of activities by which they can be achieved.}

\subsection{Main Objectives}
\emph{This refers to the overall intention for the research. 
	It is derived from the topic and statement of the problem and should be a summary in one short sentence. 
	The student should state what the study intends to accomplish in general.}

\subsection{Specific Objectives}
\emph{Specific objectives arise from the general objective.
	They focus on the variables and the relationship between them as postulated in the topic, statement of the problem and general objective. 
	The general objective is broken down into 2-3 specific areas of focus that form the specific objectives/issues of the study.
	Specific objectives are stated in short practical sentences and numbered in Arabic numerals ( 1, 2,3).
	They must be SMART i.e. Specific, Measurable, Achievable, Realistic and Time-bound.}

\section{Justification}
\emph{Explain to the reader the urgency and need for the project/ study/ research.
	Perhaps give situations that have occurred that bring out this urgency – give measures e.g. losses, deaths etc.}

\section{Scope}
\emph{Scope and project location as the village or community name(s), the nearest town and the distance and direction from the nearest town.
	A map indicating the position of the project area in relation to nearby geographical features should be included.}

\section{Conceptual Frame Work}
\emph{The conceptual frame work is a scheme of concepts or variables and the postulated relationships between them. 
	The variables are operationalised, i.e. broken down into how they are represented in the real world. 
	Operationalisation of the variables helps to give a working definition and the observable equivalents that will represent its occurrence in the real world i.e. its indicators. 
	The value of the conceptual framework is in helping the student to do a number of things through a diagram or illustration.
		\begin{itemize}
			\item The student looks at different variables (concepts) and how they relate to one another in the stated problem in her/his proposed study.
			\item Two or three of the relationships that are most relevant to the student are then picked as the specific objectives of the study.
		\end{itemize} 
	The student uses the diagram/illustration to outline the nature of relationship between the variables of the study. 
	The conceptual framework is presented diagrammatically but there must be an explanation in narrative form. 
	The student should state what the diagram/illustration is and what it shows.}