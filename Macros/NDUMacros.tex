\newenvironment{conditions}
{\par\vspace{\abovedisplayskip}\noindent\begin{tabular}{>{$}l<{$} @{${}={}$} l}}
	{\end{tabular}\par\vspace{\belowdisplayskip}}	% Parameter list of formulars

\newcommand{\ShowAuthorAandB}{%
	\AuthorA \\            		
	\texttt{\AuthorAID}
	\ifx \AuthorB \undefined \empty \else
	\and \AuthorB \\ \texttt{\AuthorBID}
	\fi
}

\newcommand{\ShowAuthorB}{%
	\ifx \AuthorB \undefined \empty \else
	\AuthorB
	\fi
}

\newcommand{\ShowAuthorBID}{%
	\ifx \AuthorBID \undefined \empty \else
	\AuthorBID
	\fi
}

\BeforeTOCHead[toc]{{\pdfbookmark[0]{\contentsname}{toc}}} 

%% The original Koma script command \maketitle has been changed
%  Sections made ineffective are between "% Comment Start" and "% Comment End"
%  Sections added are between "% Modification Start" and "% Modification End"
\makeatletter
\renewcommand*\maketitle[1][1]{%
	\expandafter\ifnum \csname scr@v@3.12\endcsname>\scr@compatibility\relax
	\else
	\def\and{%
	\end{tabular}%
	\hskip 1em \@plus.17fil%
	\begin{tabular}[t]{c}%
	}%
	\fi
	\if@titlepage
	\begin{titlepage}
		\setcounter{page}{%
			#1%
		}%
		\if@titlepageiscoverpage
		\edef\titlepage@restore{%
			\noexpand\endgroup
			\noexpand\global\noexpand\@colht\the\@colht
			\noexpand\global\noexpand\@colroom\the\@colroom
			\noexpand\global\vsize\the\vsize
			\noexpand\global\noexpand\@titlepageiscoverpagefalse
			\noexpand\let\noexpand\titlepage@restore\noexpand\relax
		}%
		\begingroup
		\topmargin=\dimexpr \coverpagetopmargin-1in\relax
		\oddsidemargin=\dimexpr \coverpageleftmargin-1in\relax
		\evensidemargin=\dimexpr \coverpageleftmargin-1in\relax
		\textwidth=\dimexpr
		\paperwidth-\coverpageleftmargin-\coverpagerightmargin\relax
		\textheight=\dimexpr
		\paperheight-\coverpagetopmargin-\coverpagebottommargin\relax
		\headheight=0pt
		\headsep=0pt
		\footskip=\baselineskip
		\@colht=\textheight
		\@colroom=\textheight
		\vsize=\textheight
		\columnwidth=\textwidth
		\hsize=\columnwidth
		\linewidth=\hsize
		\else
		\let\titlepage@restore\relax
		\fi
		\let\footnotesize\small
		\let\footnoterule\relax
		\let\footnote\thanks
		\renewcommand*\thefootnote{\@fnsymbol\c@footnote}%
		\let\@oldmakefnmark\@makefnmark
		\renewcommand*{\@makefnmark}{\rlap\@oldmakefnmark}%
		\ifx\@extratitle\@empty
		\ifx\@frontispiece\@empty
		\else
		\if@twoside\mbox{}\next@tpage\fi
		\noindent\@frontispiece\next@tdpage
		\fi
		\else
		\noindent\@extratitle
		\ifx\@frontispiece\@empty
		\else
		\next@tpage
		\noindent\@frontispiece
		\fi
		\next@tdpage
		\fi
		\setparsizes{\z@}{\z@}{\z@\@plus 1fil}\par@updaterelative
		\ifx\@titlehead\@empty \else
		\begin{minipage}[t]{\textwidth}%
			\usekomafont{titlehead}{\@titlehead\par}%
		\end{minipage}\par
		\fi
		\null\vfill
		\begin{center}
% Comment Start
%			\ifx\@subject\@empty \else
%			{\usekomafont{subject}{\@subject\par}}%
%			\vskip 3em
%			\fi
% Comment End
% Comment Start
%			{\usekomafont{title}{\huge \@title\par}}%
% Comment End
% Modification Start
			{\usekomafont{title}{\linespread{1} \huge \@title\par}}%
% Modification End
			\vskip 1em
			{\ifx\@subtitle\@empty\else\usekomafont{subtitle}{\@subtitle\par}\fi}%
			\vskip 2em
			{%
				\usekomafont{author}{%
					\lineskip 0.75em
					\begin{tabular}[t]{c}
						\@author
					\end{tabular}\par
				}%
			}%
% Modification Start
			\vskip 2em
			\ifx\@subject\@empty \else
			{\usekomafont{subject}{\@subject\par}}%
			\fi
% Modification End
% Comment Start
%			\vskip 1.5em
%			{\usekomafont{date}{\@date \par}}%
% Comment End
			\vskip \z@ \@plus3fill
			{\usekomafont{publishers}{\@publishers \par}}%
% Comment Start
%			\vskip 3em
% Comment End
% Modification Start
			\vskip 2em
			{\usekomafont{date}{\@date \par}}%
% Modification End
		\end{center}\par
		\@thanks\global\let\@thanks\@empty
% Comment Start, seems to use more space resulting in funny 2nd page with long titles
%		\vfill\null
% Comment End
		\if@twoside
		\@tempswatrue
		\expandafter\ifnum \@nameuse{scr@v@3.12}>\scr@compatibility\relax
		\else
		\ifx\@uppertitleback\@empty\ifx\@lowertitleback\@empty
		\@tempswafalse
		\fi\fi
		\fi
		\if@tempswa
		\next@tpage
		\begin{minipage}[t]{\textwidth}
			\@uppertitleback
		\end{minipage}\par
		\vfill
		\begin{minipage}[b]{\textwidth}
			\@lowertitleback
		\end{minipage}\par
		\@thanks\global\let\@thanks\@empty
		\fi
		\else
		\ifx\@uppertitleback\@empty\else
		\ClassWarning{\KOMAClassName}{%
			non empty \string\uppertitleback\space ignored
			by \string\maketitle\MessageBreak
			in `twoside=false' mode%
		}%
		\fi
		\ifx\@lowertitleback\@empty\else
		\ClassWarning{\KOMAClassName}{%
			non empty \string\lowertitleback\space ignored
			by \string\maketitle\MessageBreak
			in `twoside=false' mode%
		}%
		\fi
		\fi
		\ifx\@dedication\@empty
		\else
		\next@tdpage\null\vfill
		{\centering\usekomafont{dedication}{\@dedication \par}}%
		\vskip \z@ \@plus3fill
		\@thanks\global\let\@thanks\@empty
		\cleardoubleemptypage
		\fi
		\ifx\titlepage@restore\relax\else\clearpage\titlepage@restore\fi
	\end{titlepage}
	\else
	\par
	\@tempcnta=%
	#1%
	\relax\ifnum\@tempcnta=1\else
	\ClassWarning{\KOMAClassName}{%
		Optional argument of \string\maketitle\space ignored\MessageBreak
		in `titlepage=false' mode%
	}%
	\fi
	\ifx\@uppertitleback\@empty\else
	\ClassWarning{\KOMAClassName}{%
		non empty \string\uppertitleback\space ignored
		by \string\maketitle\MessageBreak
		in `titlepage=false' mode%
	}%
	\fi
	\ifx\@lowertitleback\@empty\else
	\ClassWarning{\KOMAClassName}{%
		non empty \string\lowertitleback\space ignored
		by \string\maketitle\MessageBreak
		in `titlepage=false' mode%
	}%
	\fi
	\begingroup
	\let\titlepage@restore\relax
	\renewcommand*\thefootnote{\@fnsymbol\c@footnote}%
	\let\@oldmakefnmark\@makefnmark
	\renewcommand*{\@makefnmark}{\rlap\@oldmakefnmark}%
	\next@tdpage
	\if@twocolumn
	\ifnum \col@number=\@ne
	\ifx\@extratitle\@empty
	\ifx\@frontispiece\@empty\else\if@twoside\mbox{}\fi\fi
	\else
	\@makeextratitle
	\fi
	\ifx\@frontispiece\@empty
	\ifx\@extratitle\@empty\else\next@tdpage\fi
	\else
	\next@tpage
	\@makefrontispiece
	\next@tdpage
	\fi
	\@maketitle
	\else
	\ifx\@extratitle\@empty
	\ifx\@frontispiece\@empty\else\if@twoside\mbox{}\fi\fi
	\else
	\twocolumn[\@makeextratitle]%
	\fi
	\ifx\@frontispiece\@empty
	\ifx\@extratitle\@empty\else\next@tdpage\fi
	\else
	\next@tpage
	\twocolumn[\@makefrontispiece]%
	\next@tdpage
	\fi
	\twocolumn[\@maketitle]%
	\fi
	\else
	\ifx\@extratitle\@empty
	\ifx\@frontispiece\@empty\else \mbox{}\fi
	\else
	\@makeextratitle
	\fi
	\ifx\@frontispiece\@empty
	\ifx\@extratitle\@empty\else\next@tdpage\fi
	\else
	\next@tpage
	\@makefrontispiece
	\next@tdpage
	\fi
	\@maketitle
	\fi
	\ifx\titlepagestyle\@empty\else\thispagestyle{\titlepagestyle}\fi
	\@thanks\global\let\@thanks\@empty
	\endgroup
	\fi
	\setcounter{footnote}{0}%
	\expandafter\ifnum \csname scr@v@3.12\endcsname>\scr@compatibility\relax
	\let\thanks\relax
	\let\maketitle\relax
	\let\@maketitle\relax
	\global\let\@thanks\@empty
	\global\let\@author\@empty
	\global\let\@date\@empty
	\global\let\@title\@empty
	\global\let\@subtitle\@empty
	\global\let\@extratitle\@empty
	\global\let\@frontispiece\@empty
	\global\let\@titlehead\@empty
	\global\let\@subject\@empty
	\global\let\@publishers\@empty
	\global\let\@uppertitleback\@empty
	\global\let\@lowertitleback\@empty
	\global\let\@dedication\@empty
	\global\let\author\relax
	\global\let\title\relax
	\global\let\extratitle\relax
	\global\let\titlehead\relax
	\global\let\subject\relax
	\global\let\publishers\relax
	\global\let\uppertitleback\relax
	\global\let\lowertitleback\relax
	\global\let\dedication\relax
	\global\let\date\relax
	\fi
	\global\let\and\relax
}%
\makeatother

%%%%
%%   PDF INFORMATION / META DATA
%%%%
\makeatletter
\hypersetup{  % provided by hyperref package
	pdfinfo={
		Title={\@title},
		Subject={\ThesisType},
		Author={\AuthorA; \ShowAuthorB},
		Copyright={Ndejje University},
		Keywords={\keywords}
	}
}
\makeatother

%%%%
%%   New command \showcalculation: 
%%   textual equation as optional argument 
%%   numbers with operators as mandatory argument
%%	
%%   Source: https://stackoverflow.com/questions/49393784/how-can-i-sum-two-numbers-in-latex-with-my-own-command/49393863
%%%%
\NewDocumentCommand{\showcalculation}{o m}{$
	\IfValueTF{#1}
	{#1}{#2} = \fpeval{#2}
	$}
