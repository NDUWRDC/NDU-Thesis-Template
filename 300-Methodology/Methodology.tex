%%
%% Chapter: 3
%%
\chapter{Methodology}
\label{cha:Methodology}      %% No special characters, no space character

%Remove explanations and start with text here
\emph{This is a detailed description of selected methodology and should be presented in unambiguous terms.
The section comprises:
\begin{itemize}
	\item Research or project design - which describes the nature and pattern the research followed e.g. whether it is historical, descriptive survey, experimental or quasi experimental and location (optional), etc.
	\item Tools and equipment used to accomplish the project. A full description of each should be	detailed.
	\item Research approaches – Qualitative or quantitative.
	\item Description of the geographical area and where population of the study exists.
	\item Description of the population from which samples have been selected.
	\item Data collection methods; including instruments and procedures used in the research or project described.
	\item Data quality control, which refers to reliability and validity of instruments.
	\item Measurements, which refer to the formulae or scales in the study.
	\item Data analysis, which involves organization and interpretation of the data generated.
	Both raw data and the analyzed form be kept both electronically and hardcopy for further reference later.
	Follow the following steps in preparing data for analysis:
	(i) receive the raw data sources
	(ii) Create electronic data base from the raw data sources
	(iii) Clean/Edit the database 
	(iv) Correct and clarify the raw data sources
	(v) Finalize database
	(vi) Create data files from the data bases.
\end{itemize}
The chapter should have ? pages.}

\section{Topic One}
\lipsum[1]
\section{Topic Two}
\lipsum[2]
\section{Topic Three}
\lipsum[3]
