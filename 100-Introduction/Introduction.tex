%%
%% Chapter: 1
%%
\chapter{Introduction}
\label{cha:Introduction}      %% No special characters, no space character

%Remove explanations and start with text here. Leave sections and subsections.
\emph{The introduction should "set the stage" for which the work is undertaken.
	It should go from the general to the specific; that is, begin with a general description of the problem or technology area and proceed to the particular issue that is being addressed.
	It is not a summary of the work but rather the background and rationale for doing it.}

\section{Background}
\emph{Background information and events to acquaint the reader with the purpose for carrying out the work (e.g. production difficulties, redesign, design problems, material selection, excessive deflections, etc.).
General references to other work (journals, text books, newspapers, etc.) that help define the problem.
This section should have a maximum of one page.}

\section{Problem Statement}
\emph{An accurate and concise statement of the problem providing details not included in the abstract that leads into the body of the report.
This section should have a maximum of half a page.}

\section{Objectives}
\emph{Main objective and the specific objectives to be achieved and scope of activities by which they can be achieved.}

\subsection{Main Objective}
\emph{This refers to the overall intention for the research.
	It is derived from the topic and statement of the problem and should be a summary in one short sentence.
	The student should state what the study intends to accomplish in general.}

\subsection{Specific Objectives}
\emph{Specific objectives arise from the general objective.
	They focus on the variables and the relationship between them as postulated in the topic, statement of the problem and general objective.
	The general objective is broken down into 2-3 specific areas of focus that form the specific objectives/issues of the study.
	Specific objectives are stated in short practical sentences and numbered in Arabic numerals (1, 2, 3).
	They must be SMART i.e. Specific, Measurable, Achievable, Realistic and Time-bound.
	Students must have not more than four objectives.}

\section{Justification}
\emph{Explain to the reader the urgency and need for the project/ study/ research.
	Perhaps give situations that have occurred that bring out this urgency – give measures e.g. losses, deaths etc.
	This section should have not more than two paragraphs of a maximum of six lines each.}

\section{Scope}
\emph{The scope of your research simply refers to the boundaries of the research.
	Physical boundaries can be like the country, district, sub-county, firm, section of a road and so on and can even be put in a table for people who are going to collect data from more than three firms or locations.
	A map indicating the position of the project area in relation to nearby geographical features should be included.
	The technical scope comes from the objectives i.e. if you have been contracted to build a structure, your technical scope may be to procure materials, supervise and pay workers, build as per the plan to finishing and then finally commissioning the structure.
	Maintenance and renovation are outside your scope of work.}
\section{Conceptual Framework}
\emph{The conceptual framework is a scheme of concepts or variables and the postulated relationships between them.
	The variables are operationalised, i.e. broken down into how they are represented in the real world.
	Operationalisation of the variables helps to give a working definition and the observable equivalents that will represent its occurrence in the real world i.e. its indicators.
	The value of the conceptual framework is in helping the student to do a number of things through a diagram or illustration.
		\begin{itemize}
			\item The student looks at different variables (concepts) and how they relate to one another in the stated problem in her/his proposed study.
			\item Two or three of the relationships that are most relevant to the student are then picked as the specific objectives of the study.
		\end{itemize}
	The conceptual framework is a diagrammatic expression that combines your research objectives, methods, and tools and also has a loop or loops to some starting point or earlier points in the flow.
	The student should state what the diagram/illustration is and what it shows in three paragraphs, not bullets, on the input, process, and output.}
